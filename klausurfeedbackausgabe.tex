\documentclass[a6paper,10pt]{scrartcl}
\usepackage{tabularx}
\usepackage[T1]{fontenc}
\usepackage[utf8]{inputenc}
\usepackage{lmodern}
\usepackage[left=1cm,right=1cm,bottom=1cm]{geometry}

\usepackage[ngerman]{babel}

\usepackage[automark,headsepline]{scrpage2}
\pagestyle{scrheadings}
\cfoot{}
\ihead{KURSNAME}
\chead{}
\ohead{Klausur vom 04.04.2018}


\pagestyle{scrheadings}

\begin{document}

\section*{Max Mustermann} \begin{tabularx}{\textwidth}{lX}
 Lernfeld(er): &Lernfeld 1, Lernfeld 2\\ 
 Klausurnote: &3+\\
 Erreichte Punkte: &54.0 von 96.0\\
 Kommentar: &Kommentar 1\end{tabularx}

 \vfill \subsection*{Notenspiegel}


\begin{tabular}{c|c|c|c|c|c}
\quad 1 \quad & \quad 2 \quad & \quad 3 \quad & \quad 4 \quad & \quad 5 \quad & \quad 6 \quad\\\hline1.0 & 0.0 & 3.0 & 1.0 & 1.0 & 0.0 \\
\end{tabular}



 \vfill LEHRERNAME, \today
 \clearpage
 
 
\section*{Timo Test} \begin{tabularx}{\textwidth}{lX}
 Lernfeld(er): &Lernfeld 1, Lernfeld 2\\ 
 Klausurnote: &3+\\
 Erreichte Punkte: &59.0 von 96.0\\
 Kommentar: &The answer is for Python. If you use tabs only, you get an 8-space indentation, unless you expand tabs to something else than 8 spaces, in which case it will look bad on other editors. If you mix tabs and spaces, it may break (see question) or look broken if you have other than 8-space expansion of tabs. In summary: Using tabs for indentation is incredibly bad. Never do that ever (except for languages/file formats that require it). The end\end{tabularx}

 \vfill \subsection*{Notenspiegel}


\begin{tabular}{c|c|c|c|c|c}
\quad 1 \quad & \quad 2 \quad & \quad 3 \quad & \quad 4 \quad & \quad 5 \quad & \quad 6 \quad\\\hline1.0 & 0.0 & 3.0 & 1.0 & 1.0 & 0.0 \\
\end{tabular}



 \vfill LEHRERNAME, \today
 \clearpage
 
 
\section*{Victor Versuch} \begin{tabularx}{\textwidth}{lX}
 Lernfeld(er): &Lernfeld 1, Lernfeld 2\\ 
 Klausurnote: &3-\\
 Erreichte Punkte: &42.0 von 96.0\\
 Kommentar: &Kommentar 3\end{tabularx}

 \vfill \subsection*{Notenspiegel}


\begin{tabular}{c|c|c|c|c|c}
\quad 1 \quad & \quad 2 \quad & \quad 3 \quad & \quad 4 \quad & \quad 5 \quad & \quad 6 \quad\\\hline1.0 & 0.0 & 3.0 & 1.0 & 1.0 & 0.0 \\
\end{tabular}



 \vfill LEHRERNAME, \today
 \clearpage
 
 
\section*{Ulf Ulfson} \begin{tabularx}{\textwidth}{lX}
 Lernfeld(er): &Lernfeld 1, Lernfeld 2\\ 
 Klausurnote: &4+\\
 Erreichte Punkte: &37.0 von 96.0\\
 Kommentar: &Kommentar 4\end{tabularx}

 \vfill \subsection*{Notenspiegel}


\begin{tabular}{c|c|c|c|c|c}
\quad 1 \quad & \quad 2 \quad & \quad 3 \quad & \quad 4 \quad & \quad 5 \quad & \quad 6 \quad\\\hline1.0 & 0.0 & 3.0 & 1.0 & 1.0 & 0.0 \\
\end{tabular}



 \vfill LEHRERNAME, \today
 \clearpage
 
 
\section*{Peter Peterson} \begin{tabularx}{\textwidth}{lX}
 Lernfeld(er): &Lernfeld 1, Lernfeld 2\\ 
 Klausurnote: &5+\\
 Erreichte Punkte: &16.0 von 96.0\\
 Kommentar: &Kommentar 5\end{tabularx}

 \vfill \subsection*{Notenspiegel}


\begin{tabular}{c|c|c|c|c|c}
\quad 1 \quad & \quad 2 \quad & \quad 3 \quad & \quad 4 \quad & \quad 5 \quad & \quad 6 \quad\\\hline1.0 & 0.0 & 3.0 & 1.0 & 1.0 & 0.0 \\
\end{tabular}



 \vfill LEHRERNAME, \today
 \clearpage
 
 
\section*{Harry Hundert-Prozent} \begin{tabularx}{\textwidth}{lX}
 Lernfeld(er): &Lernfeld 1, Lernfeld 2\\ 
 Klausurnote: &1+\\
 Erreichte Punkte: &96.0 von 96.0\\
 Kommentar: &Kommentar 6\end{tabularx}

 \vfill \subsection*{Notenspiegel}


\begin{tabular}{c|c|c|c|c|c}
\quad 1 \quad & \quad 2 \quad & \quad 3 \quad & \quad 4 \quad & \quad 5 \quad & \quad 6 \quad\\\hline1.0 & 0.0 & 3.0 & 1.0 & 1.0 & 0.0 \\
\end{tabular}



 \vfill LEHRERNAME, \today
 \clearpage
 
 
\section*{Nils Nulpe} \begin{tabularx}{\textwidth}{lX}
 Lernfeld(er): &Lernfeld 1, Lernfeld 2\\ 
 Klausurnote: &6\\
 Erreichte Punkte: &0.0 von 96.0\\
 Kommentar: &Kommentar 7\end{tabularx}

 \vfill \subsection*{Notenspiegel}


\begin{tabular}{c|c|c|c|c|c}
\quad 1 \quad & \quad 2 \quad & \quad 3 \quad & \quad 4 \quad & \quad 5 \quad & \quad 6 \quad\\\hline1.0 & 0.0 & 3.0 & 1.0 & 1.0 & 0.0 \\
\end{tabular}



 \vfill LEHRERNAME, \today
 \clearpage
 
 
\section*{Peter Pleite} \begin{tabularx}{\textwidth}{lX}
 Lernfeld(er): &Lernfeld 1, Lernfeld 2\\ 
 Klausurnote: &6\\
 Erreichte Punkte: &0.0 von 96.0\\
 Kommentar: &Kein Kommentar\end{tabularx}

 \vfill \subsection*{Notenspiegel}


\begin{tabular}{c|c|c|c|c|c}
\quad 1 \quad & \quad 2 \quad & \quad 3 \quad & \quad 4 \quad & \quad 5 \quad & \quad 6 \quad\\\hline1.0 & 0.0 & 3.0 & 1.0 & 1.0 & 0.0 \\
\end{tabular}



 \vfill LEHRERNAME, \today
 \clearpage
 
 
\end{document}