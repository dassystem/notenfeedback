\documentclass[a4paper]{scrartcl}
\usepackage{todonotes}
\usepackage{lastpage}

\usepackage[left=2.5cm,right=2cm, top=2.5cm, bottom=3cm]{geometry}
\usepackage[
	fach=Name:,
	lerngruppe={},
	typ=LF 1.3.6,
%	debug,
	%klausurtyp=klausur,
	%klausurtyp=klasse,
	%klausurtyp=kurs,
%	nummer=2,
	farbig,
%	datumAnzeigen,
%	namensfeldAnzeigen,
%	seitenzahlen=immer,
	%zitate=guillemets,
	%zitate=quotes,
	loesungen=seite,
	%loesungen=folgend,
	%erwartungshorizontAnzeigen,
%	erwartungshorizontStil=standard,
%	kmkPunkte,
	module={
		Kuerzel,
		Papiertypen,
		Symbole,
		Texte
	},
]{schule}

\usepackage{weva}

\title{}
\author{author}
\date{12.08.2016}

\newcommand*{\titleGP}{\begingroup % Create the command for including the title page in the document
\centering
	
	\rule{\textwidth}{1.6pt}\vspace*{-\baselineskip}\vspace*{2pt} % Thick horizontal line
	\rule{\textwidth}{0.4pt}\\[\baselineskip] % Thin horizontal line
	
	{\LARGE Einrichtung}\\
	{\large Studiengang}
	
	\rule{\textwidth}{0.4pt}\\[\baselineskip] % Thin horizontal line
	
	 % Title
	
	\begin{tabular}{rl}
		{\LARGE \underline{Klausurdeckblatt}}&\\
		Kurs & 35\\
		Theorieblock & 3\\
		Klausurdatum & 13.01.2017\\
		Lernfeld & 1.3.6\\
		Dozent & LEHRERNAME\\
	\end{tabular}
	
	\rule{\textwidth}{0.4pt}\vspace*{-\baselineskip}\vspace{3.2pt} % Thin horizontal line
	\rule{\textwidth}{1.6pt}\\[\baselineskip] % Thick horizontal line

	\vspace{1em}
		\begin{tabular}{rl}
			\textbf{Vom Prüfling auszufüllen:}&\\\\
			%Vorname & \luecke{0.4\linewidth}\\\\
			Vor- und Nachname & \luecke{0.4\linewidth}\\\\
			
			%Hiermit erkläre ich, dass ich prüfungsfähig bin.&\\
			%Die Klausur wurde von mir auf Vollständigkeit  & \\
			%überprüft und eigenständig verfasst. & \\
			%Ich habe die allgemeinen Hinweise zur  & \\
			%Bearbeitung gelesen und verstanden. & \luecke{0.4\linewidth}\\
			%& Datum, Unterschrift\\\\
		\end{tabular}
	
	\rule{\textwidth}{0.4pt}\vspace*{-\baselineskip}\vspace{3.2pt} % Thin horizontal line

\vspace{2em}
	
	\begin{tabular}{rl}
		\textbf{Vom Prüfer auszufüllen:}&\\\\
		 \textbf{Erreichte Punkte} & \luecke{0.4\linewidth}\\\\
		 \textbf{Note} & \luecke{0.4\linewidth}\\\\
		\luecke{0.4\linewidth} & \\
		Datum, Unterschrift&\\\\
		
	\end{tabular}
	
\rule{\textwidth}{0.4pt}\vspace*{-\baselineskip}\vspace{3.2pt} % Thin horizontal line
	
	\paragraph*{Allgemeine Hinweise zur Bearbeitung:}
	
	\begin{itemize}
		\item Die Höchstpunktzahl betragt \totalpoints. 
		\item Die Klausur besteht aus \pageref*{LastPage} Seiten inklusive des Deckblattes.
		\item Auf jeder Seite ist oben links der Name einzutragen.
		\item Hilfsmittel sind nicht erlaubt. Handys und andere elektronische Geräte sind nicht erlaubt.
		\item Alle Aufgaben sind leserlich zu lösen. Nicht eindeutige Lösungen werden nicht gewertet.
		\item Täuschungsversuche haben das Nichtbestehen der Klausur zur Folge.%, ohne dass vorher ermahnt wird.
		%\item der Klausurraum darf nicht verlassen werden, es dein denn, es liegt ein ärztliches Attest vor, dass die Notwendigkeit bescheinigt. Unerlaubtes Verlassen des Raumes führt zur sofortigen Beendigung der Klausur für diese Teilnehmerin/ diesen Teilnehmer.
		
	\end{itemize}
	
	\endgroup}

\begin{document}
%\pagestyle{empty}
\titleGP

%\pagestyle{plain}

\newpage
\section*{Aufgaben} Für alle Aufgaben zusammen gibt es \totalpoints .\\
Diese Klausur enthält verschiedene Arten von \textbf{Multiple Choice Aufgaben}, sowohl mit Einfachnennung und mit Mehrfachnennungen. Bei denen mit Mehrfachnennungen sind mehrere Antworten korrekt. Für \underline{richtig} gesetzte Kreuze gibt es je einen Punkt, für \underline{falsch} gesetzte Kreuze gibt es einen Punkt Abzug. In einer Aufgabe können keine negativen Punkte erreicht werden.

\begin{aufgabe}{10}
	Nennen Sie die 5 lokalen Entzündungszeichen mit ihren deutschen Begriffen und den lateinischen Begriffen.  
		
	\begin{center}
		\begin{tabular}{ccc}
			\luecke{.4\textwidth} & $-$ & \luecke{.4\textwidth}  \\ \\ 
			\luecke{.4\textwidth} & $-$ & \luecke{.4\textwidth}  \\ \\
			\luecke{.4\textwidth} & $-$ & \luecke{.4\textwidth}  \\ \\
			\luecke{.4\textwidth} & $-$ & \luecke{.4\textwidth}  \\ \\ 
			\luecke{.4\textwidth} & $-$ & \luecke{.4\textwidth}  \\ \\ 
		\end{tabular}
	\end{center}

	\begin{loesung}
		\begin{itemize}
			\item Rötung - rubor
			\item Überwärmung - calor
			\item Schwellung - tumor
			\item Schmerz - dolor
			\item Eingeschränkte Bewegung - functio laesa
		\end{itemize}
	\end{loesung}
\end{aufgabe}

\begin{aufgabe}{3}
	Nennen Sie die 3 keimreduzierenden Verfahren für Oberflächen.
	\begin{center}
		
		\begin{enumerate}
			\item \luecke{.9\textwidth}
			\item \luecke{.9\textwidth}
			\item \luecke{.9\textwidth}
		\end{enumerate}
	\end{center}
	
	\begin{loesung}
		\begin{itemize}
			\item Reinigung
			\item Desinfektion
			\item Sterilisation
		\end{itemize}
	\end{loesung}
\end{aufgabe}

\begin{aufgabe}{1}
	Kreuzen Sie die korrekte Antwortmöglichkeit an. Worauf ist bei der Verwendung von Desinfektionsmitteln zu achten?
	\begin{center}
		\begin{mcumgebung}(1)
			\choice![\mcrichtig] Verwendung gelisteter Mittel, Wirkspektrum, Konzentration und Einwirkzeit
			\choice! Wirkspektrum, Einwirkzeit, ---- und Bewohnerfreundlichkeit
			\choice! Sterile Lagerung, Ansetzen der Lösung jede Stunde, Verwendung von Alkoholen und -----
			\choice! -----
		\end{mcumgebung}
	\end{center}
	
	\begin{loesung}
		\begin{mcumgebung}(1)
			\choice![\mcrichtig] Verwendung gelisteter Mittel, Wirkspektrum, Konzentration und Einwirkzeit
			\choice! Wirkspektrum, Einwirkzeit, ---- und Bewohnerfreundlichkeit
			\choice! Sterile Lagerung, Ansetzen der Lösung jede Stunde, Verwendung von Alkoholen und -----
			\choice! -----
		\end{mcumgebung}
	\end{loesung}
\end{aufgabe}



\begin{aufgabe}{1}
	Kreuzen Sie die korrekte Antwortmöglichkeit an. Einrichtungen des Gesundheitswesens benötigen ein Hygienekonzept. Was geht der Erstellung des Konzeptes voraus?
	\begin{center}
		\begin{mcumgebung}(1)
			\choice! Juristische Prüfung der Notwendigkeit eines Hygienekonzeptes
			\choice![\mcrichtig] Eine Risikobewertung nach Erreger, Risiko für Bewohner/Personal, Maßnahmen und Praktikabilität
			\choice! Schulung der Bewohner und deren Angehörigen über die chirurgische Händedesinfektion
			\choice! Eine Risikobewertung unter finanziellen und wettbewerbstechnischen Gesichtspunkten
		\end{mcumgebung}
	\end{center}
	
	\begin{loesung}
		\begin{mcumgebung}(1)
			\choice! Juristische Prüfung der Notwendigkeit eines Hygienekonzeptes
			\choice![\mcrichtig] Eine Risikobewertung nach Erreger, Risiko für Bewohner/Personal, Maßnahmen und Praktikabilität
			\choice! Schulung der Bewohner und deren Angehörigen über die Händedesinfektion
			\choice! Eine Risikobewertung unter finanziellen und wettbewerbstechnischen Gesichtspunkten
		\end{mcumgebung}
	\end{loesung}
\end{aufgabe}



\begin{aufgabe}{2}
	Ordnen Sie die Antwortmöglichkeiten der beiden Listen einander zu und kreuzen Sie die korrekte Antwortmöglichkeit an.
	
	\vspace{1em}
	\begin{tabular}{ll}
		(1) Reinigung  & (A) Abtötung aller vermehrungsfähigen Mikroorganismen\\
		(2) Sterilisation & (B) Entfernung von sichtbaren Schmutz\\
		(3) Desinfektion & (C) Abtötung, Reduzierung und Inaktivierung von Mikroorganismen \\
	\end{tabular}
	
	
	\begin{mcumgebung}(1)
		\choice![\mcrichtig] 1B, 2A, 3C
		\choice! 1B, 2C, 3A
		\choice! 1A, 2B, 3C
		\choice! 1C, 2A, 3B
	\end{mcumgebung}
	
	\begin{loesung}
		\begin{mcumgebung}(1)
			\choice![\mcrichtig] 1B, 2A, 3C
			\choice! 1B, 2C, 3A
			\choice! 1A, 2B, 3C
			\choice! 1C, 2A, 3B
		\end{mcumgebung}
	\end{loesung}
\end{aufgabe}






%\begin{aufgabe}{3}
%	Nennen Sie 3 \textbf{Kriterien}, die im Umgang mit Handschuhen zu beachten sind.
%	\begin{enumerate}
%		\item \luecke{.9\textwidth}
%		\item \luecke{.9\textwidth}
%		\item \luecke{.9\textwidth}
%	\end{enumerate}
	
%	\begin{loesung}
%		\begin{itemize}
%			\item Keine Desinfektion
%			\item HD vor dem Rausnehmen der HS
%			\item HD nach dem Ausziehen
%			\item Kontaminationsfreies Ausziehen (Technik)
%			\item Nur kurz Tragen --> Feuchtarbeit
%			\item Nicht in der Kitteltasche tragen
%			\item Patientenbezogen verwenden
%			\item Wechsel bei Kontamination
%			\item Kein Schutz --> Mikroperforationen
%			\item Einteilung in sterile, unsterile und chemikalienbeständige HS
%			\item Indikationen zum Tragen
%				\begin{itemize}
%					\item um Übertragung zu vermeiden, wenn Kontakt mit Sekreten, Exkreten oder infektiösen Material zu erwarten ist
%					\item Injektionen, BE und Wundversorgung
%					\item Umgang mit benutzen Instrumenten
%					\item Pflege von inkontinenten Bewohnern
%					\item Entsorgung und Transport von pot. infektiösen Material/Abfällen
%					\item Reinigung und Desinfektion von kontaminierten Flächen
%				\end{itemize}
%		\end{itemize}
		
%	\end{loesung}
%\end{aufgabe}


\begin{aufgabe}{4}
	Kreuzen Sie die korrekten Antwortmöglichkeiten an (\underline{Mehrfachnennungen}). Welche Kriterien sind im Umgang mit Handschuhen zu beachten?
	\begin{center}
		\begin{mcumgebung}(1)
			\choice![\mcrichtig] Handschuhe werden nicht desinfiziert
			\choice! Handschuhe stellen einen Ersatz für die hygienische Händedesinfektion dar
			\choice![\mcrichtig] Handschuhe stellen keinen Ersatz für die hygienische Händedesinfektion dar
			\choice![\mcrichtig] Handschuhe werden kontaminationsfrei ausgezogen
			\choice! Handschuhe werden zimmerbezogen getragen
			\choice! Handschuhe werden in der Kitteltasche gelagert
			\choice![\mcrichtig] Handschuhe werden bei Kontamination gewechselt
			\choice! Handschuhe werden bei jeder Tätigkeit getragen
		\end{mcumgebung}
	\end{center}
	
	\begin{loesung}
		\begin{mcumgebung}(1)
			\choice![\mcrichtig] Handschuhe werden nicht desinfiziert
			\choice! Handschuhe stellen einen Ersatz für die hygienische Händedesinfektion dar
			\choice![\mcrichtig] Handschuhe stellen keinen Ersatz für die hygienische Händedesinfektion dar
			\choice![\mcrichtig] Handschuhe werden kontaminationsfrei ausgezogen
			\choice! Handschuhe werden zimmerbezogen getragen
			\choice! Handschuhe werden in der Kitteltasche gelagert
			\choice![\mcrichtig] Handschuhe werden bei Kontamination gewechselt
			\choice! Handschuhe werden bei jeder Tätigkeit getragen
		\end{mcumgebung}
	\end{loesung}
\end{aufgabe}



\begin{aufgabe}{4}
	Nennen Sie die 4 Säulen der Händehygiene.
	\begin{enumerate}
		\item \luecke{.9\textwidth}
		\item \luecke{.9\textwidth}
		\item \luecke{.9\textwidth}
		\item \luecke{.9\textwidth}
	\end{enumerate}
	
	\begin{loesung}
		\begin{itemize}
			\item Händereinigung/-waschung
			\item Händedesinfektion
			\item Hautpflege und -schutz
			\item Tragen von Handschuhen
		\end{itemize}
		
	\end{loesung}
\end{aufgabe}




\begin{aufgabe}{5}
	Frau Sommer ist 83 Jahre alt und lebt in der stationären Pflegeeinrichtung in der Sie arbeiten. Sie ist aufgrund von Arthrose sehr immobil. Sie unterhalten sich in Ihrem Frühdienst mit Frau Sommer und erklären Ihr, wann die hygienische Händedesinfektion zu erfolgen hat. Dabei verweisen Sie auf die Aktion saubere Hände, die das Modell der WHO für Indikationen der hygienischen Händedesinfektion auf den Kontext der Pflegeeinrichtung angepasst hat. Nennen Sie die 5 Indikationen der hygienischen Händedesinfektion bei immobilen Bewohnern.
	\begin{itemize}
		\item \luecke{.9\textwidth}
		\item \luecke{.9\textwidth}
		\item \luecke{.9\textwidth}
		\item \luecke{.9\textwidth}
		\item \luecke{.9\textwidth}
	\end{itemize}
	
	\begin{loesung}
		\begin{itemize}
			\item Vor Bewohnerkontakt
			\item Vor aseptischen Tätigkeiten
			\item Nach Bewohnerkontakt
			\item Nach Kontakt mit potenziell infektiösem Material
			\item Nach Kontakt mit der unmittelbaren Bewohnerumgebung
		\end{itemize}
		
	\end{loesung}
\end{aufgabe}



\begin{aufgabe}{2}
	Ordnen Sie die Antwortmöglichkeiten der beiden Listen einander zu und kreuzen Sie die korrekte Antwortmöglichkeit an.
	
	\vspace{1em}
	\begin{tabular}{ll}
		(1) Transiente Flora & (A) Fähigkeit, in anderen Organismen Infektionen auszulösen\\
		(2) Epidemiologie & (B) Lehre von der körpereigenen Abwehr\\
		(3) Immunologie & (C) Bereitschaft zur aktiven Mitarbeit\\
		(4) Residente Flora & (D) Anflugflora\\
		(5) Compliance & (E) Standortflora\\
		(6) Pathogenität & (F) Wissenschaft von der Ausbreitung von Krankheiten\\
	\end{tabular}
	
	
	\begin{mcumgebung}(1)
		\choice! 1D, 2F, 3E, 4A, 5B, 6C
		\choice! 1C, 2A, 3B, 4D, 5E, 6F
		\choice![\mcrichtig] 1D, 2F, 3B, 4E, 5C, 6A
		\choice! 1A, 2B, 3C, 4D, 5E, 6F
		\choice!1E, 2F, 3D, 4C, 5B, 6A
	\end{mcumgebung}
	
	\begin{loesung}
		\begin{mcumgebung}(1)
		\choice! 1D, 2F, 3E, 4A, 5B, 6C
		\choice! 1C, 2A, 3B, 4D, 5E, 6F
		\choice![\mcrichtig] 1D, 2F, 3B, 4E, 5C, 6A
		\choice! 1A, 2B, 3C, 4D, 5E, 6F
		\choice!1E, 2F, 3D, 4C, 5B, 6A
		\end{mcumgebung}
	\end{loesung}
\end{aufgabe}




\begin{aufgabe}{6}
	Kreuzen Sie die korrekten Antwortmöglichkeiten an (\underline{Mehrfachnennungen}). Welche Maßnahmen zählen zu der Standardhygiene?
	\begin{center}
		\begin{mcumgebung}(1)
			\choice! Mund- und Zahnpflege
			\choice! Sterilisation von Oberflächen
			\choice![\mcrichtig] Händehygiene 
			\choice![\mcrichtig] Hustenetikette
			\choice! 
			\choice![\mcrichtig] Persönliche Schutzausrüstung
			\choice!
			\choice!
			\choice![\mcrichtig] Flächenreinigung und -desinfektion
			\choice!
			\choice![\mcrichtig] Sterilisation von Instrumenten
			\choice![\mcrichtig] Verwendung von Sicherheitsmaterial
		\end{mcumgebung}
	\end{center}
	
	\begin{loesung}
		\begin{mcumgebung}(1)
			\choice! Mund- und Zahnpflege
			\choice! Sterilisation von Oberflächen
			\choice![\mcrichtig] Händehygiene 
			\choice![\mcrichtig] Hustenetikette
			\choice! 
			\choice![\mcrichtig] Persönliche Schutzausrüstung
			\choice!
			\choice!
			\choice![\mcrichtig] Flächenreinigung und -desinfektion
			\choice!
			\choice![\mcrichtig] Sterilisation von Instrumenten
			\choice![\mcrichtig] Verwendung von Sicherheitsmaterial
		\end{mcumgebung}
	\end{loesung}
\end{aufgabe}






\begin{aufgabe}{2}
	Ordnen Sie die Antwortmöglichkeiten der beiden Listen einander zu und kreuzen Sie die korrekte Antwortmöglichkeit an.
	
	\vspace{1em}
	\begin{tabular}{ll}
		(1) Kolonisation  & (A) Eindringen, Vermehren und Auslösen einer Abwehrreaktion\\
		(2) Infektion & (B) Verunreinigung von Gegenständen/Körpern mit Mikroorganismen\\
		(3) Kontamination & (C) Besiedeln der Haut, offener Wunden ohne Krankheitssymptome\\
	\end{tabular}
	
	
	\begin{mcumgebung}(1)
		\choice! 1C, 2B, 3A
		\choice! 1B, 2A, 3C
		\choice! 1A, 2B, 3C
		\choice![\mcrichtig] 1C, 2A, 3B
	\end{mcumgebung}
	
	\begin{loesung}
		\begin{mcumgebung}(1)
		\choice! 1C, 2B, 3A
		\choice! 1B, 2A, 3C
		\choice! 1A, 2B, 3C
		\choice![\mcrichtig] 1C, 2A, 3B
		\end{mcumgebung}
	\end{loesung}
\end{aufgabe}


\begin{aufgabe}{5}
	Nennen Sie die 5 Erregergruppen.
	\begin{center}
		\begin{enumerate}
			\item \luecke{.9\textwidth}
			\item \luecke{.9\textwidth}
			\item \luecke{.9\textwidth}
			\item \luecke{.9\textwidth}
			\item \luecke{.9\textwidth}
		\end{enumerate}
	\end{center}
	
	\begin{loesung}
		\begin{enumerate}
			\item Bakterien
			\item Viren
			\item Pilze
			\item Parasiten
			\item Prionen
		\end{enumerate}
	\end{loesung}
\end{aufgabe}



\begin{aufgabe}{3}
	Kreuzen Sie die korrekten Antwortmöglichkeiten an (\underline{Mehrfachnennungen}). Welche aussagen treffen auf Viren zu?
	\begin{center}
		\begin{mcumgebung}(1)
			\choice![\mcrichtig] Sie besitzen keinen eigenen Stoffwechsel
			\choice! Sind identisch mit Bakterien und Parasiten
			\choice![\mcrichtig] Benötigen zur Vermehrung eine Wirtszelle
			\choice![\mcrichtig] Es gibt behüllte und unbehüllte Viren
			\choice! Behüllte Viren sind viel widerstandsfähiger als unbehüllte Viren
			\choice! Viren sind immer deutlich größer als Bakterien
		\end{mcumgebung}
	\end{center}
	
	\begin{loesung}
		\begin{mcumgebung}(1)
			\choice![\mcrichtig] Sie besitzen keinen eigenen Stoffwechsel
			\choice! Sind identisch mit Bakterien und Parasiten
			\choice![\mcrichtig] Benötigen zur Vermehrung eine Wirtszelle
			\choice![\mcrichtig] Es gibt behüllte und unbehüllte Viren
			\choice! Behüllte Viren sind viel widerstandsfähiger als unbehüllte Viren
			\choice! Viren sind immer deutlich größer als Bakterien
		\end{mcumgebung}
	\end{loesung}
\end{aufgabe}




\begin{aufgabe}{2}
	Ordnen Sie die Antwortmöglichkeiten der beiden Listen einander zu und kreuzen Sie die korrekte Antwortmöglichkeit an.
	
	\vspace{1em}
	\begin{tabular}{ll}
		(1) Erreger aus der Umgebung & (A) Über Zwischenwirte\\
		(2) Von Mensch zu Mensch & (B) Künstliche Eintrittspforte\\
		(3) Gefäßzugänge & (C) Diaplazentare Übertragung\\
		(4) Indirekte Übertragung & (D) Exogene Infektion \\
		(5) Während der Schwangerschaft & (E) Direkte Übertragung\\
	\end{tabular}
	
	
	\begin{mcumgebung}(1)
		\choice! 1C, 2B, 3A, 4D, 5E
		\choice! 1B, 2E, 3D, 4C, 5A
		\choice! 1E, 2D, 3C, 4B, 5A
		\choice! 1A, 2D, 3B, 4C, 5E
		\choice![\mcrichtig] 1D, 2E, 3B, 4A, 5C
	\end{mcumgebung}
	
	\begin{loesung}
		\begin{mcumgebung}(1)
		\choice! 1C, 2B, 3A, 4D, 5E
		\choice! 1B, 2E, 3D, 4C, 5A
		\choice! 1E, 2D, 3C, 4B, 5A
		\choice! 1A, 2D, 3B, 4C, 5E
		\choice![\mcrichtig] 1D, 2E, 3B, 4A, 5C
		\end{mcumgebung}
	\end{loesung}
\end{aufgabe}




\begin{aufgabe}{1}
	Kreuzen Sie die korrekte Antwortmöglichkeit an. Die Infektionskette besteht aus welchen Schritten?
	\begin{center}
		\begin{mcumgebung}(1)
			\choice! Infektionsquelle, Phasenverlauf, Pflege und Betreuung
			\choice!  -----, Pflege und Betreuung
			\choice!
			\choice![\mcrichtig] Infektionsquelle, Übertragungsweg, Eintrittspforte und Empfänger
		\end{mcumgebung}
	\end{center}
	
	\begin{loesung}
		\begin{mcumgebung}(1)
			\choice! Infektionsquelle, Phasenverlauf, Pflege und Betreuung
			\choice!  -----, Pflege und Betreuung
			\choice!
			\choice![\mcrichtig] Infektionsquelle, Übertragungsweg, Eintrittspforte und Empfänger
		\end{mcumgebung}
	\end{loesung}
\end{aufgabe}


\end{document}